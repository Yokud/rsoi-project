\chapter{Аналитический раздел}

\section{Существующие аналоги}

На данный момент существует некоторое количество сервисов, нацеленных на аренды автомобилей, но они имеют ряд недостатков. Данные о сравнении сервисов сведены в таблице 1.

\begin{table}[H]
	\caption{Обзор существующих аналогов}
	\begin{center}
		\begin{tabular}{| C{3 cm} | C{4 cm} | C{2.25 cm} |} 
			\hline
			
			\textbf{Название} & \textbf{Архитектура} & \textbf{Наличие отзывов} \\  
			
			\hline
			
			rentcars.ru & Микросервисная & Нет \\
			
			\hline
			
			prostoprokat.ru  & Монолитная & Нет \\
			
			\hline
			
			avtomaxi.ru & Монолитная & Есть \\
			
			\hline
		\end{tabular}
	\end{center}
\end{table}

Не все представленные сервисы имеют возможность оставлять отзывы на арендуемые автомобили, из-за чего арендатор не может узнать об актуальном состянии арендуемого автомобиля.

Также не все представленные сервисы основаны на микросервисной архитектуре, из-за чего возникают сложности с масштабированием, обслуживанием и внесением изменений в функциональность.

\section{Описание системы}

Разрабатываемый сервис должен представлять собой распределенную систему для бронирования автомобилей. Если пользователь хочет забронировать выбранный автомобиль, ему необходимо зарегистрироваться, указав информацию: фамилия, имя, отчество, номер телефона, электронная почта, пароль. После регистрации для активации аккаунта необходимо подтвердить почту путем перехода по ссылке, присланной на почту. В случае, если зарегистрированному ранее пользователю нужно получить информацию о свободных для бронирования автомобилей и выбрать один из них, ему нужно авторизоваться. Для неавторизованных пользователей доступен только просмотр списка свободных для бронирования автомобилей.

На рисунке \ref{img:pic/general} отображена схема предметной области.

\includeimage
{pic/general}
{f}
{H}
{0.8\textwidth}
{Схема предметной области.}


\section{Общие требования к системе}
Требования к системе следующие.
\begin{enumerate}
	\item Разрабатываемое ПО должно поддерживать функционирование системы в режиме 24 часов, 7 дней в неделю, 365 дней в году (24/7/365) со среднегодовым временем доступности не менее 99.9\%. Допустимое время, в течении которого система недоступна, за год должна составлять $24\cdot365\cdot0.001=8.76$ ч.
	
	\item Время восстановления системы после сбоя не должно превышать 15 минут.
	
	\item Каждый узел должен автоматически восстанавливаться после сбоя.
	
	\item Система должна поддерживать возможность <<горячего>> переконфигурирования системы. Необходимо предусмотреть поддержку добавления нового узла во время работы системы без рестарта.
	
	\item Обеспечить безопасность работоспособности за счет отказоустойчивости узлов.
\end{enumerate}

\section{Требования к функциональным характеристикам}
\begin{enumerate}
	\item По результатам работы модуля сбора статистики медиана времени отклика системы на запросы пользователя на получение информации не должна превышать 3 секунд.
	
	\item По результатам работы модуля сбора статистики медиана времени отклика системы на запросы, добавляющие или изменяющие информацию на портале не должна превышать 7 секунд.
	
	\item Система должна обеспечивать возможность запуска в современных браузерах: не менее 85\% пользователей Интернета должны пользоваться ей без какой-либо деградации функционала.
\end{enumerate}

\section{Функциональные требования к системе с точки зрения пользователя}
Система должна обеспечивать реализацию следующих функций.
\begin{enumerate}
	\item Регистрация и авторизация пользователей с валидацией вводимых данных (процесс регистрации включает в себя подтверждение адреса электронной почты).
	
	\item Аутентификация пользователей (в том числе двойная, на усмотрение пользователя).
	
	\item Разделение всех пользователей на три роли:
	\begin{itemize}
		\item Неавторизированный пользователь (гость));
		
		\item Авторизированный пользователь (клиент));
		
		\item Администратор.
	\end{itemize}
	
	\item Предоставление возможностей \textbf{Гостю, Клиенту, Администратору} представленных в таблице \ref{tbl:user-func}.
\end{enumerate}

\begin{longtable}{|p{0.5cm}|p{15.5cm}|}
	\caption{Функции пользователей}
	\label{tbl:user-func} \\
	\hline
	
	\begin{rotatebox}[origin=r]{90}
		{ \textbf{Гость}}
	\end{rotatebox} 
	& 
	1. просмотр списка доступных автомобилей; \newline
	2. получение информации о выбранном автомобиле; \newline
	3. просмотр отзывов на выбранный автомобиль; \newline
	4. регистрация в системе; \newline
	5. авторизация в системе; \\
	\hline
	
	\begin{rotatebox}[origin=r]{90}
		{ \textbf{Клиент}}
	\end{rotatebox} 
	& 
	1. функции гостя; \newline
	2. создание отзыва на ранее забронированный автомобиль; \newline
	3. просмотр собственной истории бронирования автомобилей; \newline
	4. бронирование выбранного автомобиля на указанный период; \newline
	5. оплата аренды автомобиля; \newline
	6. завершение и отмена аренды автомобиля. \\
	\hline
	
	\begin{rotatebox}[origin=r]{90}
		{ \textbf{Администратор}}
	\end{rotatebox} 
	& 
	1. функции клиента; \newline
	2. просмотр информации об арендах автомобилей указанного пользователя; \newline
	3. создание, редактирование и удалении информации о бронируемых автомобилях; \newline
	4. редактирование данных и удаление пользователей; \newline
	5. отслеживание транзакций. \\
	\hline
\end{longtable}

\section{Входные данные}
Входные параметры системы представлены в таблице \ref{tbl:input}.

\begin{longtable}{|p{3cm}|p{13cm}|}
	\caption{Входные данные}
	\label{tbl:input} \\
	\hline
	
	\textbf{Сущность} & \textbf{Входные данные} \\
	\hline
	\endfirsthead
	
	\hline
	\textbf{Сущность} & \textbf{Входные данные} \\
	\hline
	\endhead
	
	\hline
	\multicolumn{2}{c}{\textit{Продолжение на следующей странице}}
	\endfoot
	\hline
	\endlastfoot
	
	Клиент / Администратор
	&
	1. \textit{фамилия, имя} и \textit{отчество} не более 256 символов каждое поле; \newline
	2. \textit{дата рождения} в формате д/м/гггг; \newline
	3. \textit{логин} не менее 10 символов и не более 128; \newline
	4. \textit{пароль} не менее 8 символов и не более 128, как минимум одна заглавная и одна строчная буква, только латинские буквы, без пробелов, как минимум одна цифра; \newline
	5. \textit{номер телефона}; \newline
	6. \textit{электронная почта}; \newline
	7. \textit{роль} (CLIENT, ADMIN); \\
	\hline
	
	Автомобиль
	& 
	1. \textit{бренд} не более 80 символов; \newline
	2. \textit{модель} не более 80 символов; \newline
	3. \textit{регистрационный номер} не более 20 символов; \newline
	4. \textit{мощность}; \newline
	5. \textit{цена}; \newline
	6. \textit{тип} (SEDAN, SUV, MINIVAN, ROADSTER); \newline
	7. \textit{наличие}; \\
	\hline
	
	Аренда
	& 
	1. \textit{имя пользователя}; \newline
	2. \textit{дата начала аренды}; \newline
	3. \textit{дата конца аренды}; \newline
	4. \textit{статус} (IN\_PROGRESS, FINISHED, CANCELED); \\
	\hline
	
	Платеж
	&
	1. \textit{статус} (PAID, CANCELED); \newline
	2. \textit{цена}; \\
	\hline
	
	Отзыв
	& 
	1. \textit{имя пользователя}; \newline
	2. \textit{оценка}; \newline
	3. \textit{дата публикации} ; \newline
	4. \textit{отзыв} не более 500 символов.
\end{longtable}


\section{Выходные параметры}
Выходными параметрами системы являются web-страницы. В зависимости от запроса и текущей роли пользователя они содержат следующую информацию (таблица \ref{tbl:output-data}).

\begin{longtable}{|p{0.5cm}|p{15.5cm}|}
	\caption{Выходные параметры}
	\label{tbl:output-data} \\
	\hline
	
	\begin{rotatebox}[origin=r]{90}
		{ \textbf{Гость}}
	\end{rotatebox} 
	& 
	1. Список доступных автомобилей, указывается: \newline
	• \textit{бренд}; \newline
	• \textit{модель}; \newline
	• \textit{регион}; \newline
	• \textit{цена}; \newline
	• \textit{тип}; \newline
	• \textit{рейтинг}; \\
	\cline{2-2}
	
	&
	2. Подробное описание выбранного автомобиля: \newline
	• \textit{бренд}; \newline
	• \textit{модель}; \newline
	• \textit{регистрационный номер}; \newline
	• \textit{мощность}; \newline
	• \textit{цена}; \newline
	• \textit{тип}; \newline
	• \textit{рейтинг}; \\
	\cline{2-2}
	
	&
	3. Отзывы выбранного автомобиля: \newline
	• \textit{имя пользователя}; \newline
	• \textit{оценка}; \newline
	• \textit{дата}; \newline
	• \textit{отзыв}; \\
	\cline{2-2}
	
	&
	4. Общая информация о сайте; \\
	\cline{2-2}
	
	&
	5. Контактная информация поддержки;
	\\
	\hline
	
	\begin{rotatebox}[origin=r]{90}
		{ \textbf{Клиент}}
	\end{rotatebox} 
	& 
	1. Список доступных автомобилей, указывается: \newline
	• \textit{бренд}; \newline
	• \textit{модель}; \newline
	• \textit{регион}; \newline
	• \textit{цена}; \newline
	• \textit{тип}; \newline
	• \textit{рейтинг}; \\
	\cline{2-2}
	
	&
	2. Подробное описание выбранного автомобиля: \newline
	• \textit{бренд}; \newline
	• \textit{модель}; \newline
	• \textit{регистрационный номер}; \newline
	• \textit{мощность}; \newline
	• \textit{цена}; \newline
	• \textit{тип}; \newline
	• \textit{рейтинг}; \\
	\cline{2-2}
	
	&
	3. Параметры аренды выбранного автомобиля:
	• \textit{дата начала аренды}; \newline
	• \textit{дата конца аренды}; \newline
	• \textit{суммарная стоимость аренды}; \\
	\cline{2-2}
	
	&
	4. Отзывы выбранного автомобиля: \newline
	• \textit{имя пользователя}; \newline
	• \textit{оценка}; \newline
	• \textit{дата}; \newline
	• \textit{отзыв}; \\
	\cline{2-2}
	
	&
	5. Общая информация о сайте; \\
	\cline{2-2}
	
	&
	6. Контактная информация поддержки; \\
	\cline{2-2}
	
	&
	7. История бронирования автомобилей:
	• \textit{дата начала аренды}; \newline
	• \textit{дата конца аренды}; \newline
	• \textit{суммарная стоимость аренды}; \newline
	• \textit{статус аренды};
	\\
	\hline
	
	\begin{rotatebox}[origin=r]{90}
		{ \textbf{Администратор}}
	\end{rotatebox} 
	& 
	1. Список автомобилей, указывается: \newline
	• \textit{бренд}; \newline
	• \textit{модель}; \newline
	• \textit{регион}; \newline
	• \textit{цена}; \newline
	• \textit{тип}; \newline
	• \textit{рейтинг}; \newline
	• \textit{наличие}; \\
	\cline{2-2}
	
	&
	2. Подробное описание выбранного автомобиля: \newline
	• \textit{бренд}; \newline
	• \textit{модель}; \newline
	• \textit{регистрационный номер}; \newline
	• \textit{мощность}; \newline
	• \textit{цена}; \newline
	• \textit{тип}; \newline
	• \textit{рейтинг}; \\
	\cline{2-2}
	
	&
	3. Отзывы выбранного автомобиля: \newline
	• \textit{имя пользователя}; \newline
	• \textit{оценка}; \newline
	• \textit{дата}; \newline
	• \textit{отзыв}; \\
	\cline{2-2}
	
	&
	4. Общая информация о сайте; \\
	\cline{2-2}
	
	&
	5. Контактная информация поддержки; \\
	\cline{2-2}
	
	&
	6. Список зарегистированных пользователей:
	• \textit{фамилия, имя} и \textit{отчество}; \newline
	• \textit{дата рождения}; \newline
	• \textit{номер телефона}; \newline
	• \textit{электронная почта}; \\
	\cline{2-2}
	
	&
	7. История бронирования автомобилей указанного пользователя:
	• \textit{дата начала аренды}; \newline
	• \textit{дата конца аренды}; \newline
	• \textit{суммарная стоимость аренды}; \newline
	• \textit{статус аренды}; \\
	\cline{2-2}
	
	&
	8. Список всех пользователей с возможностью их удаления или ограничения доступа на время, а также возможность редактирования информации о пользователе в соответствии с полями из пункта 6; \\
	\cline{2-2}
	
	&
	9. Список всех автомобилей для бронирования с возможностью их удаления, а также возможность редактирования информации об автомобиле в соответствии с полями из пункта 2; \\
	\cline{2-2}
	
	&
	10. Статистика по порталу, собранная через сервис статистики.
	\\
	\hline
\end{longtable}

\section{Топология Системы}
На рисунке \ref{img:pic/topology} изображен один из возможных вариантов топологии разрабатываемой распределенной Системы.

\includeimage
{pic/topology}
{f}
{H}
{\textwidth}
{Топология системы.}

Система будет состоять из фронтенда и 8 подсистем:
\begin{itemize}
	\item сервис-координатор;
	
	\item сервис регистрации и авторизации;
	
	\item сервис-координатор автомобилей;
	
	\item сервис автомобилей;
	
	\item сервис аренд;
	
	\item сервис оплат;
	
	\item сервис отзывов;
	
	\item сервис статистики.
\end{itemize}
%
\textbf{Фротенд} принимает запросы от пользователей по протоколу HTTP и анализирует их. На основе проведенного анализа выполняет запросы к микросервисам бекенда, агрегирует ответы и отсылает их пользователю. \\
\textbf{Сервис-координатор} -- сервис, ответственный за координацию запросов внутри системы. Для реализации балансировки запросов используется инфраструктура Kubernetes. \\
\textbf{Сервис-регистрации и авторизации} отвечает за:
\begin{itemize}
	\item возможность регистрации нового клиента;
	
	\item аутентификацию пользователя (клиента и администратора);
	
	\item авторизацию пользователя;
	
	\item двухфазную авторизацию;
	
	\item подтверждение почты;
	
	\item восстановление пароля к странице;
	
	\item выход из сессии.
\end{itemize}

Сервис регистрации и авторазации в своей работе использует базу данных, которая хранит следующую инфомацию:

\begin{itemize}
	\item Пользователь
	\begin{itemize}
		\item \textit{идентификатор};
		\item \textit{фамилия, имя} и \textit{отчество};
		\item \textit{дата рождения};
		\item \textit{логин};
		\item \textit{пароль};
		\item \textit{номер телефона};
		\item \textit{электронная почта};
		\item \textit{роль}.
	\end{itemize}
\end{itemize}

\textbf{Сервис-координатор автомобилей} -- сервис, ответственный за координацию запросов к соответствующим сервисам автомобилей, каждый из которых привязан к своему региону по регистрационному номеру. То есть, если, например, для поиска был выбрана московская область, то сервис определяет код региона для этого субъекта и перенаправляет запрос к тому сервису, ответственному за этот регион.

Сервис использует в своей работе базу данных, в которой хранится следующая информация:

\begin{itemize}
	\item Адрес сервиса:
	\begin{itemize}
		\item \textit{идентификатор};
		\item \textit{название региона}
		\item \textit{код региона};
		\item \textit{ip-адрес}.
	\end{itemize}
\end{itemize}

\textbf{Сервис автомобилей} реализует функции:
\begin{itemize}
	\item получение списка автомобилей, как доступных для бронирования, так и занятых;
	
	\item получение информации об автомобиле;
	
	\item создание записи о новом автомобиле;
	
	\item редактирование информации об автомобиле;
	
	\item удаление автомобиля.
\end{itemize}

Сервис автомобилей в своей работе использует базу данных, которая хранит следующую инфомацию:

\begin{itemize}
	\item Автомобиль
	\begin{itemize}
		\item \textit{идентификатор};
		\item \textit{бренд};
		\item \textit{модель};
		\item \textit{регистрационный номер};
		\item \textit{мощность};
		\item \textit{цена};
		\item \textit{тип};
		\item \textit{наличие}.
	\end{itemize}
\end{itemize}

\textbf{Сервис аренд} реализует функции:
\begin{itemize}
	\item получение списка аренд указанного пользователя;
	
	\item получение информации по конкретной аренде пользователя;
	
	\item создание записи о бронировании автомобиля;
	
	\item редактирование записи о бронировании автомобиля;
	
	\item завершение аренды автомобиля;
	
	\item отмена аренды автомобиля;
	
	\item удаление записи о бронировании автомобиля.
\end{itemize}

Сервис аренд в своей работе использует базу данных, которая хранит следующую инфомацию:

\begin{itemize}
	\item Аренда
	\begin{itemize}
		\item \textit{идентификатор};
		\item \textit{имя пользователя};
		\item \textit{идентификатор} соответствующей \textit{оплаты};
		\item \textit{идентификатор} соответствующего \textit{автомобиля};
		\item \textit{дата начала аренды};
		\item \textit{дата конца аренды};
		\item \textit{статус}.
	\end{itemize}
\end{itemize}

\textbf{Сервис оплат} реализует функции:
\begin{itemize}
	\item получение списка оплат;
	
	\item получение информации о конкретной оплате;
	
	\item создание записи о новой оплате;
	
	\item отмена платежа;
	
	\item редактирование информации об оплате;
	
	\item удаление оплаты.
\end{itemize}

Сервис оплат в своей работе использует базу данных, которая хранит следующую инфомацию:

\begin{itemize}
	\item Оплата
	\begin{itemize}
		\item \textit{идентификатор};
		\item \textit{статус};
		\item \textit{цена}.
	\end{itemize}
\end{itemize}

\textbf{Сервис отзывов} реализует функции:
\begin{itemize}
	\item получение списка всех отзывов;
	\item получение списка личных отзывов пользователя;
	\item получение отзывов о конкретном автомобиле;
	\item изменение отзыва;
	\item удаление отзыва;
	\item получение выбранного отзыва.
\end{itemize}

Сервис отзывов в своей работе использует базу данных, которая хранит следующую инфомацию:

\begin{itemize}
	\item Отзыв
	\begin{itemize}
		\item \textit{идентификатор};
		\item \textit{идентификатор} соответствующего \textit{автомобиля};
		\item \textit{идентификатор} соответствующего \textit{пользователя};
		\item \textit{числовая оценка} от 1 до 5;
		\item \textit{дата публикации};
		\item \textit{отзыв}.
	\end{itemize}
\end{itemize}

\textbf{Сервис статистики} отвечает за логирование событий во всей системе для осуществления возможности быстрого детектирования, локализации и воспроизведения ошибки в случае её возникновения.

\section{Требования к программной реализации}
\begin{enumerate}
	\item Требуется использовать архитектуру SPA (Single Page Application) для реализации системы. Использование CSS обязательно.
	
	\item Требуется создать сервис, выполняющий функцию Identity Provider~\cite{idprovider}, реализовать протокол OpenID Connect.
	
	\item  Система состоит из микросервисов. Каждый микросервис отвечает за свою область логики работы приложения и должны быть запущены изолированно друг от друга (один сервис -- один docker-контейнер~\cite{docker}).
	
	\item  При необходимости, каждый сервис имеет своё собственное хранилище, запросы между базами запрещены.
	
	\item  Каждый сервис при получении запроса выполняет валидацию JWT токена с помощью JWKs, которые он получает из Identity Provider.
	
	\item  При разработке базы данных необходимо учитывать, что доступ к ней должен осуществляться по протоколу TCP.
	
	\item  Для межсервисного взаимодействия использовать HTTP (придерживаться RESTful).
	
	\item  Выделить Gateway Service как единую точку входа и межсервисной коммуникации для исключения горизонтальных запросов между сервисами.
	
	\item  При недоступности систем портала должна осуществляться деградация функционала или выдача пользователю сообщения об ошибке.
	
	\item  Необходимо предусмотреть авторизацию пользователей, как через интерфейс приложения, так и через приложения двухфазной авторизации.
	
	\item  Валидация входных данных должна производиться и на стороне пользователя с помощью TypeScript скриптов, и на стороне фронтенда. Бекенды не должны валидировать входные данные, так как пользователь не может к ним обращаться напрямую, бекенды должны получать уже отфильтрованные входные данные от фронтенда.
	
	\item  Для запросов, выполняющих обновление данных на нескольких узлах распределенной системы, в случае недоступности одной из систем, необходимо выполнять полный откат транзакции.
	
	\item  На сервисе-координаторе для всех операций чтения реализовать паттерн Circuit Breaker. Накапливать статистику в памяти, и если система не ответила $N$ раз, то в $N+1$ раз вместо запроса сразу отдавать fallback. Через небольшой timeout выполнить запрос к реальной системе, чтобы проверить ее состояние.
	
	\item  В случае недоступности данных из некритичного источника (не основного), возвращается fallback-ответ. В зависимости от ситуации, это может быть:
	\begin{itemize}
		\item пустой объект или массив;
		\item объект, с заполненным полем (uid или подобным), по которому
		идет связь с другой системой;
		\item строка по умолчанию (если при этом не меняется тип переменной).
	\end{itemize}
	
	\item  Код хранить на Github, для сборки использовать Github Actions.
\end{enumerate}

\section{Функциональные требования к подсистемам}

\textbf{Фронтенд} -- серверное  приложение, предоставляет пользовательский интерфейс и внешний API системы, при  разработке которого нужно учитывать следующее:
\begin{itemize}
	\item должен  принимать  запросы  по  протоколу  HTTP и формировать ответы пользователям в формате HTML;
	
	\item в зависимости от типа запроса должен отправлять последовательные запросы в соответствующие микросервисы;
	
	\item запросы к микросервисам необходимо осуществлять по протоколу HTTP;
	
	\item данные необходимо передавать в формате JSON.
\end{itemize}

\textbf{Сервис-координатор} -- серверное приложение, через которое проходит весь поток запросов и ответов, должен соответствовать следующим требованиям разработки:
\begin{itemize}
	\item обрабатывать запросы в соответствии со своим назначением, описанным в топологии системы;
	
	\item принимать и возвращать данные в формате JSON по протоколу HTTP;
	
	\item накапливать статистику запросов, в случае, если система не ответила $N$ раз, то в $N + 1$ раз вместо запроса сразу отдавать fallback. Через некоторое время выполнить запрос к реальной системе, чтобы проверить ее состояние;
	
	\item выполнять проверку существования клиента, также регистрацию и аутентификацию пользователей;
	
	\item при получении запроса выполнять валидацию JWT токена с помощью JWKs, которые он получает из Identity Provider;
	
	\item осуществлять деградацию функциональности в случае отказа некритического сервиса (зависит от семантики запроса);
	
	\item существовать в нескольких экземплярах, чтобы координация запросов не была узким местом приложения;
	
	\item уведомлять сервис статистики о событиях в системем.
\end{itemize}

\textbf{Сервис регистрации и авторизации, сервис-координатор автомобилей, сервис автомобилей, сервис аренд, оплат и сервис отзывов } -- это серверные приложения, которые должны отвечать следующим требованиям по разработке:
\begin{itemize}
	\item обрабатывать запросы в соответствии со своим назначением, описанным в топологии системы;
	
	\item принимать и возвращать данные в формате JSON по протоколу HTTP;
	
	\item осуществлять доступ к СУБД по протоколу TCP.
\end{itemize}

\textbf{Сервис статистики} -- это серверное приложение, которое должно отвечать следующим требованиям по разработке:
\begin{itemize}
	\item обрабатывать запросы в соответствии со своим назначением, описанным в топологии системы;
	
	\item принимать и возвращать данные в формате JSON по протоколу HTTP.
\end{itemize}