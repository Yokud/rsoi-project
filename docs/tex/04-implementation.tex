\chapter{Технологический раздел}

\section{Выбор операционной системы}

Согласно требованиям технического задания, разрабатываемый портал должен обладать высокой доступностью, работать на типичных архитектурах ЭВМ (Intel x86, Intel x64), а так же быть экономически недорогим для сопровождения. Таким образом, можно сформулировать следующие требования к операционной системе:

\begin{itemize}
	\item Распространенность. На рынке труда должно быть много специалистов, способных администрировать распределенную систему, работающую под управлением выбранной операционной системы. Windows, будучи самой распространенной операционной системой для настольных и серверных компьютеров, полностью отвечает этому требованию. Найти квалифицированных системных администраторов с опытом работы на Windows Server не составит труда.
	
	\item Надежность. Операционная система должна широко использоваться в стабильных проектах, таких как Mail.Ru, Vk.com, Google.com. Эти компании обеспечивают высокую работоспособность своих сервисов, и на их опыт можно положиться. Windows Server зарекомендовала себя как надежная и стабильная платформа, используемая множеством крупных компаний, включая Amazon, Microsoft и другие, обеспечивающие высокую доступность своих сервисов.
	
	\item Наличие требуемого программного обеспечения. Выбор операционной системы не должен ограничивать разработчиков в выборе программного обеспечения, библиотек. Windows обладает обширной экосистемой программного обеспечения, среди которого большинство популярных языков программирования, фреймворков и баз данных имеют версии для Windows.
	
	\item Цена. Windows Server является коммерческим продуктом и требует приобретения лицензии. Однако, Microsoft предлагает различные варианты лицензирования, позволяющие подобрать оптимальное соотношение цены и функционала в зависимости от масштаба проекта. 
\end{itemize}

Windows Server~\cite{windows-server} может стать подходящим выбором для реализации портала, соответствуя требованиям высокой доступности, распространенности, наличию необходимых инструментов разработки и оптимальной стоимости владения.

\section{Выбор СУБД}

В качестве СУБД была выбрана PostgreSQL~\cite{postgresql}, так как она наилучшим образом подходит под требования разрабатываемой системы:

\begin{itemize}
	\item Масштабируемость: PostgreSQL поддерживает горизонтальное масштабирование, что позволяет распределить данные и запросы между несколькими узлами базы данных. Это особенно полезно в географически распределенных системах, где данные и пользователи могут быть разбросаны по разным регионам.
	
	\item Географическая репликация: PostgreSQL предоставляет возможность настройки репликации данных между различными узлами базы данных, расположенными в разных географических зонах. Это позволяет обеспечить отказоустойчивость и более быстрый доступ к данным для пользователей из разных частей нашей страны.
	
	\item Гибкость и функциональность: PostgreSQL обладает широким набором функций и возможностей, что делает его подходящим для различных типов приложений и использования в распределенной среде. Он поддерживает сложные запросы, транзакции, хранимые процедуры и многое другое.
	
	\item Надежность и отказоустойчивость: PostgreSQL известен своей надежностью и стабильностью работы. В распределенной географической системе это особенно важно, поскольку он способен обеспечить сохранность данных и доступность даже при сбоях в отдельных узлах.
\end{itemize}

\section{Выбор языка разработки и фреймворков компонент портала}

Исходя из требований к системе, можно обосновать выбор языка программирования C\#~\cite{csharp} и фреймворка ASP.NET Core~\cite{aspnet} для разработки портала следующими аргументами:

\begin{enumerate}
	\item Язык программирования C\#:
	\begin{itemize}
		\item Кроссплатформенность: C\# -- современный объектно-ориентированный язык программирования, работающий на платформе .NET. Благодаря .NET Core, приложения на C\# могут запускаться на различных операционных системах, включая Linux, что обеспечивает совместимость с выбранной ОС.
		
		\item Интеграция с PostgreSQL: C\# и .NET предоставляют богатый набор инструментов для работы с базами данных, включая PostgreSQL. Библиотеки, такие как Npgsql, обеспечивают удобное и эффективное взаимодействие с PostgreSQL, что соответствует техническому заданию.
		
		\item Производительность: C\# компилируется в промежуточный язык (IL), который затем исполняется виртуальной машиной .NET. Это обеспечивает высокую производительность, сравнимую с компилируемыми языками, такими как C++.
		
		\item Расширяемость: C\# поддерживает объектно-ориентированный подход, что упрощает разработку сложных и масштабируемых систем. Богатая экосистема библиотек .NET предоставляет готовые решения для множества задач, сокращая время разработки и повышая надежность проекта.
	\end{itemize}
	\item Фреймворк ASP.NET Core:
	\begin{itemize}
		\item Высокая производительность: ASP.NET Core - это быстрый и легковесный фреймворк, оптимизированный для обработки веб-запросов. Он обеспечивает высокую пропускную способность и быстрый отклик, что важно для производительности портала.
		
		\item Встроенная поддержка многопоточности: ASP.NET Core эффективно использует многопоточность для обработки большого количества запросов одновременно. Это обеспечивает масштабируемость и отзывчивость портала, даже при высокой нагрузке.
		
		\item Создание RESTful API: ASP.NET Core предоставляет мощные инструменты для разработки RESTful API, что упрощает интеграцию с другими системами и создание мобильных приложений.
		
		\item Модульность и расширяемость: ASP.NET Core построен на основе модульной архитектуры, позволяющей легко расширять функциональность приложения. Благодаря широкому выбору библиотек и компонентов .NET, ASP.NET Core обеспечивает гибкость и масштабируемость для реализации различных требований проекта.
	\end{itemize}
\end{enumerate}

Таким образом, выбор C\# и ASP.NET Core для разработки портала обоснован высокой производительностью, совместимостью с выбранными технологиями, расширяемостью, удобством разработки и богатой экосистемой инструментов.