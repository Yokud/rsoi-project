\chapter*{ЗАКЛЮЧЕНИЕ}
\addcontentsline{toc}{chapter}{ЗАКЛЮЧЕНИЕ}

В рамках данной курсовой работы была проведена разработка вебприложения для аренды автомобилей, включающая как клиентскую, так и серверную части. На основании проведённого анализа современных технологий были выбраны оптимальные инструменты для создания высокопроизводительного и масштабируемого решения. В качестве бэкенд-фреймворка был использован ASP.NET языка программирования C\#, обеспечивающий высокую производительность, поддержку асинхронного программирования и простоту интеграции с современными инструментами. Для фронтенда был выбран React, отличающийся гибкостью, большим сообществом и поддержкой современных подходов к созданию интерактивных пользовательских интерфейсов.

Разработанное приложение предоставляет удобный интерфейс для поиска и аренды автомобилей, а также обеспечивает автоматизированный сбор и обработку данных о пользователях и аренде. В процессе реализации были решены задачи по организации архитектуры приложения, обеспечению безопасности данных.
