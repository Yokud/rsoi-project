\chapter{Конструкторский раздел}

\section{Концептуальный дизайн}

Для создания функциональной модели портала, отражающей его основные функции и потоки информации наиболее наглядно использовать нотацию
IDEF0. На рисунке \ref{img:idef0/01_A0} приведена концептуальная модель системы. На рисунке \ref{img:idef0/02_A0} представлена декомпозиция функциональной модели системы.

\includeimage
{idef0/01_A0}
{f}
{H}
{\textwidth}
{Концептуальная модуль системы в нотации IDEF0}

\includeimage
{idef0/02_A0}
{f}
{H}
{\textwidth}
{Детализированная концептуальная модель системы в нотации IDEF0}

\section{Сценарии функционирования системы}

\subsection*{Регистрация пользователя}

\begin{enumerate}
	\item Пользователь переходит на страницу регистрации с помощью кнопки <<Регистрация>>, либо автоматически перенаправляется на страницу авторизации, где ему будет предложено перейти к регистрации, при попытке совершения действий, которые невозможно совершить без регистрации (например, забронировать автомобиль). После чего пользователь будет перенаправлен на страницу регистрации.
	
	\item Пользователь заполняет различные наборы полей на странице регистрации. Валидация входных данных осуществляется <<на лету>> на стороне пользователя. При отправке данных на фронтенд, он тоже производит валидацию.
	
	\item Пользователь нажимает кнопку <<Регистрация>>, после чего пользователь перенаправляется на главную страницу портала.
\end{enumerate}

\subsection*{Авторизация на портале пользователя}

\begin{enumerate}
	\item Пользователь переходит на страницу авторизации с помощью кнопки <<Авторизация>>, либо автоматически перенаправляется на соответствующую страницу при попытке совершения действий, которые невозможно совершить без регистрации (например, забронировать автомобиль).
	
	\item Вводит учётные данные, нажимает кнопку <<Войти>>.
	
	\item Пользователь даёт согласие на использование его данных. Если пользователь не дает согласия, то он перенаправляется на страницу с ошибкой.
	
	\item Пользователь перенаправляется на главную страницу портала.
\end{enumerate}

\subsection*{Просмотр доступных для бронирования автомобилей}

Сценарий доступен как для авторизованного (Клиент), так и для неавторизованного (Гость) пользователя.

\begin{enumerate}
	\item При переходе на страницу <<Аренда авто>> пользователю на экране предоставляется список доступных для аренды автомобилей.
	
	\item Пользователь задает конкретезированные и диапазонные параметры поиска (бренд, регион, цена, тип) и после нажатия кнопки <<Обновить>> получает список доступных для аренды автомобилей в соответствии с заданными им фильтрами просмотра.
\end{enumerate}

\subsection*{Аренда автомобиля}

\begin{enumerate}
	\item При переходе на страницу <<Аренда авто>> пользователю на экране предоставляется список доступных для аренды автомобилей.
	
	\item Пользователь задает конкретезированные и диапазонные параметры поиска (бренд, регион, мощность, цена, тип) и после нажатия кнопки <<Обновить>> получает список доступных для аренды автомобилей в соответствии с заданными им фильтрами просмотра.
	
	\item Пользователь выбирает автомобиль и переходит на страницу с детальной информацией об автомобиле.
	
	\item Нажав кнопку <<Арендовать>>, пользователь должен выбрать срок аренды (дата начала и окончания аренды), после чего пользователю предоставляется информация о сумме оплаты за аренду.
	
	\item Пользователь нажимает кнопку <<Оплатить>>, после чего происходит операция оплаты (денежная транзакция).
	
	\item В случае успешной оплаты, пользователь арендует автомобиль, т. е. помечает автомобиль как занятый, а информация об аренде попадает в историю аренд пользователя.
	
	\item В случае завершения аренды, пользователь оставляет оценку автомобилю и опционально заполняет отзыв об автомобиле. В случае отмены аренды, необходимо указать причину.
\end{enumerate}

\subsection*{Создание объявления об аренде автомобиля}

Сценарий доступен только пользователю с ролью <<Администратор>>.

\begin{enumerate}
	\item При переходе на страницу <<Аренда авто>> пользователю на экране предоставляется список доступных для аренды автомобилей.
	
	\item Пользователь переходит на страницу создания объявления с помощью кнопки <<Создать новое объявление>>.
	
	\item Пользователь вводит данные о питомце (бренд, модель, регистрационный номер, мощность, цена, тип) в соответствующие поля, которые валидируются <<на лету>> на стороне пользователя.
	
	\item Пользователь переходит на страницу для прикрепления фотографий автомобиля с помощью кнопки <<Далее>>.
	
	\item Пользователь прикрепляет фотографии автомобиля в соответствии с требованиями к формату. После чего происходит проверка регистрационных номеров в ГИБДД.
	
	\item Пользователь перенаправляется на страницу созданного им объявления, где помимо информации по объявлению отображается статус ожидания проверки регистрационных номеров и место в очереди на проверку. После успешной проверки объявление публикуется для общего доступа.
\end{enumerate}

\subsection*{Получение статистики}

Сценарий доступен только пользователю с ролью <<Администратор>>.

\begin{enumerate}
	\item Пользователь нажимает на кнопку <<Посмотреть историю запросов>> и перенаправляется на соответствующую страницу.
	
	\item Пользователь нажимает на кнопку <<Получить статистику>>.
	
	\item Пользователь перенаправляется на страницу просмотра статистики о запросах.
\end{enumerate}

\section{Диаграммы прецедентов}

Для детальной разработки портала используется унифицированный язык моделирования UML. В системе выделены три роли: Гость, Клиент, Администратор. На рисунках \ref{img:use_case/guest_uc}-\ref{img:use_case/admin_uc} представлены диаграммы прецедентов для каждой из ролей.

\includeimage
{use_case/guest_uc}
{f}
{H}
{0.85\textwidth}
{Диаграмма с точки зрения Гостя.}

\includeimage
{use_case/client_uc}
{f}
{H}
{0.85\textwidth}
{Диаграмма с точки зрения Клиента.}


\includeimage
{use_case/admin_uc}
{f}
{H}
{0.85\textwidth}
{Диаграмма с точки зрения Администратора.}

\section{Спецификация классов}

Иерархии классов для разработки серверных приложений представлены в виде диаграммы классов:

\begin{itemize}
	\item сервис автомобилей (Рисунок \ref{img:uml/Carscontroller});
	
	\item сервис аренд (Рисунок \ref{img:uml/Rentalcontroller});
	
	\item сервис оплат (Рисунок \ref{img:uml/Paymentcontroller});
	
	\item сервис-координатор (Рисунок \ref{img:uml/Gatewaycontroller}).
\end{itemize}

\includeimage
{uml/Carscontroller}
{f}
{H}
{\textwidth}
{Диаграмма классов сервиса автомобилей}

\includeimage
{uml/Rentalcontroller}
{f}
{H}
{\textwidth}
{Диаграмма классов сервиса аренд}

\includeimage
{uml/Paymentcontroller}
{f}
{H}
{\textwidth}
{Диаграмма классов сервиса оплат}

\includeimage
{uml/Gatewaycontroller}
{f}
{H}
{\textwidth}
{Диаграмма классов сервиса-координатора}

\subsection*{Описание основных классов сервисов}

Сервисы автомобилей, аренд, оплат и отзывов спроектированы похожим образом. Они имеют:

\begin{itemize}
	\item слой доступа к данным, реализованный с помощью паттерна <<Репозиторий>> для абстракции хранения и для эффективного использования активных подключений к базе данных;
	
	\item слой логики работы сервиса;
	
	\item слой интерфейса (контроллер) -- слой связи с другими сервисами с помощью http-запросов.
\end{itemize}

Классы контроллеров описывают набор HTTP-методов, которые доступны на сервисе. Фактически занимается распаковкой данных, пришедших по сети, заполнением необходимых структур этими данными и вызовом функций, реализующих логику работы методов. Также занимаются подготовкой результирующих данных к отправке по сети после выполнения запроса.

\subsection*{Описание классов сервиса автомобилей}

Car -- класс, описываюий автомобиль, имеет поля:

\begin{itemize}
	\item Идентификатор автомобиля;
	
	\item Бренд;
	
	\item Модель;
	
	\item Регистрационный номер;
	
	\item Мощность;
	
	\item Цена за суточную аренду;
	
	\item Тип автомобиля;
	
	\item Доступность для аренды;
\end{itemize}

CarInfo -- класс, содержащий краткую информацию об автомобиле в виде, пригодном для пользователя.

CarResponse -- класс, содержащий полную информацию об автомобиле в виде, пригодном для пользователя.

PaginationResponse -- класс для пагинации автомобилей.

CarsController -- контроллер, описывающий набор HTTP-методов, который доступен на сервисе автомобилей.

ICarsRepository -- абстрактный интерфейс для работы с данными автомобилей (интерфейс спроектирован в соответствии с паттерном <<репозиторий>>). CarsRepository -- реализация этого абстрактного интерфейса для работы с данными, хранящимися под управлением СУБД Postgres.

ICarsService -- абстрактный интерфейс, содержащий слой логики работы с данными. CarsService -- реализация этого абстрактного интерфейса.

\subsection*{Описание классов сервиса аренд}

Rental -- класс, описывающий запись об аренде автомобиля, имеет поля:

\begin{itemize}
	\item Идентификатор аренды;
	
	\item Идентификатор пользователя;
	
	\item Идентификатор оплаты;
	
	\item Идентификатор автомобиля;
	
	\item Дата начала аренды;
	
	\item Дата конца аренды;
	
	\item Статус аренды;
\end{itemize}

CreatePaidRentalRequest -- класс, описывающий запрос на создание записи об аренде после её оплаты.

RentalController -- контроллер, описывающий набор HTTP-методов, который доступен на сервисе аренд.

IRentalRepository -- абстрактный интерфейс для работы с данными аренд (интерфейс спроектирован в соответствии с паттерном <<репозиторий>>). RentalRepository -- реализация этого абстрактного интерфейса для работы с данными, хранящимися под управлением СУБД Postgres.

IRentalService -- абстрактный интерфейс, содержащий слой логики работы с данными. RentalService -- реализация этого абстрактного интерфейса.

\subsection*{Описание классов сервиса оплат}

Payment -- класс, описывающий информацию о платеже, имеет поля:

\begin{itemize}
	\item Идентификатор оплаты;
	
	\item Статус платежа;
	
	\item Сумма оплаты;
\end{itemize}

PaymentInfo -- класс, содержащий информацию о платеже в виде, пригодном для пользователя.

PaymentController -- контроллер, описывающий набор HTTP-методов, который доступен на сервисе платежей.

IPaymentRepository -- абстрактный интерфейс для работы с данными платежей (интерфейс спроектирован в соответствии с паттерном <<репозиторий>>). PaymentRepository -- реализация этого абстрактного интерфейса для работы с данными, хранящимися под управлением СУБД Postgres.

IPaymentService -- абстрактный интерфейс, содержащий слой логики работы с данными. PaymentService -- реализация этого абстрактного интерфейса.

\subsection*{Описание классов сервиса-координатора}

GatewayController -- класс, описывающий набор HTTP-методов, которые доступны на сервисе. Фактически представляет собой весь программный интерфейс системы. В своей работе для обслуживания приходящих запросов сервис использует все вышеописанные интерфейсы сервисов, а также сервис перенаправляет статистику запросов в очередь Kafka~\cite{kafka} с помощью интерфейса KafkaStatistcsService.
